\chapter{Introduzione}

\section{Cause}

\index{Cause}

Al giorno d'oggi la costruzione di reti progettate per la condivisione ad alta capacit� di dati e il fatto che la comunicazione e la diffusione di concetti viene maggiormente veicolata attraverso l'uso di computer e di tecnologie simili, hanno permesso lo scambio sempre pi� crescente di dati governativi e commerciali mettendo, per�, in luce un problema fondamentale: la molteplicit� di lingue e di dizionari di dati.\\
Questo problema comporta un blocco del contenuto delle informazioni esistenti e connesse ad Internet e la mancanza di una lingua franca digitale � evidente nel dominio del cibo, poich� i materiali viaggiano dalla loro provenienza, agricola o meno, ai consumatori attraverso catene di processo e di distribuzione.\\
Un vocabolario bene definito e gerarchico, con relazioni logiche, ossia un'ontologia, � necessario per aiutare ad armonizzare quei dati che abbracciano i settori della sicurezza, qualit�, produzione e distribuzione alimentare e, di conseguenza, della salute e convenienza dei consumatori.

\section{FoodOn}

\index{FoodOn}