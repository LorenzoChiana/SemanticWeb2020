\chapter{Introduzione}

\section{Motivazione}

\index{Motivazione}

Al giorno d'oggi l'innovazione digitale sta influenzando molti aspetti sanitari ed economici della produzione, distribuzione e consumo degli alimenti.
Ad esempio l'Internet of Things (IoT) propone sempre pi� una visione di ambienti come quelli agricoli e industriali pervasi di sensori connessi costantemente alla rete, generando dati che possono essere utilizzati per aumentare la qualit� del prodotto e del processo stesso.
Nello specifico, in ambito agricolo e gastronomico, i dati generati da queste tecnologie possono essere utilizzati per aumentare la qualit� degli alimenti e garantire la tracciabilit�, riducendo al contempo costi, consumi e sprechi legati alle risorse.\\
Altro esempio le questioni legate alla sicurezza degli alimenti, all'autenticit� e ai conflitti derivanti dalla protezione dei marchi e tutta la logistica sull'approvvigionamento e sulla distribuzione degli alimenti locali contro multinazionali possono e vengono analizzate anche con l'aiuto di set di dati e modelli specifici per gli alimenti.\\
Questo scambio di dati sempre pi� crescente ha messo in luce un problema fondamentale: pochi sottodomini terminologici sono stati standardizzati un po' in tutti i settori.
La pletora di dizionari alimentari mantiene invisibili le informazioni sugli alimenti a causa della mancanza di interoperabilit�, influenzando di conseguenza la tracciabilit� degli alimenti, dei patogeni di origine alimentare, dei contaminanti e, pi� in generale, della qualit� degli alimenti stessi.
In passato si � tentato di creare vocabolari alimentari applicabili a livello internazionale, ma ostacoli sia tecnici che linguistici hanno scoraggiato un deposito globale per la catalogazione degli alimenti regionali e della loro composizione.\\
%la costruzione di reti progettate per la condivisione ad alta capacit� di dati e il fatto che la comunicazione e la diffusione di concetti viene maggiormente veicolata attraverso l'uso di computer e di tecnologie simili, hanno permesso lo scambio sempre pi� crescente di dati governativi e commerciali mettendo, per�, in luce un problema fondamentale: la molteplicit� di lingue e di dizionari di dati.\\
%Questo problema comporta un blocco del contenuto delle informazioni esistenti e connesse ad Internet e la mancanza di una lingua franca digitale � evidente nel dominio del cibo, poich� i materiali viaggiano dalla loro provenienza, agricola o meno, ai consumatori attraverso catene di processo e di distribuzione.\\
La combinazione dell'attuale infrastruttura Internet e dei progressi del web semantico rende ora interessanti soluzioni di tipo ontologiche.
L'ontologia fornisce una teoria formale per un dominio di indagine che specifica il significato dei termini all'interno di un vocabolario e consiste in una struttura tassonomica gerarchica e in dichiarazioni (detti assiomi) su come le entit� all'interno di un dominio sono correlate.
Termini appropriati possono essere identificati e distinti da etichette ontologiche e sinonimi che includono nomi in multilingua o nomi specifici della regione, nonch� identificatori e definizioni univoci accessibili a livello globale, evitando cos� l'uso di vocabolari ambigui.

\section{Ontologie e cibo}

\index{Ontologie e cibo}

Le ontologie sono inoltre in grado di accogliere pi� gerarchie, spesso sotto forma di tassonomie, che possono agire con sfaccettature a maggior o minor livello di dettaglio mentre si naviga tra le gerarchie stesse.
Un prodotto alimentare, ad esempio, pu� essere collegato a varie categorie di prodotti alimentari nazionali o internazionali in una sfaccettatura ``tipo prodotto", cos� come gli ingredienti per mezzo di una sfaccettatura gerarchica di animali e/o piante ``fonte di cibo".\\
Le ontologie ben progettate permettono di riutilizzare termini di altre ontologie gi� consolidate al fine di eliminare duplicati.
Ci� permette l'integrazione di ontologie altrimenti disparate tra domini.
Inoltre un'ontologia consente di migliorare un termine del vocabolario attraverso assiomi logici che un calcolatore pu� leggere e su cui pu� ragionare.

\section{Il progetto FoodOn}

\index{Il progetto FoodOn}

\href{https://foodon.org}{FoodOn} � un progetto guidato dal consorzio Hsiao Lab per costruire un'ontologia globale \textit{farm-to-fork}, ossia un vocabolario per la descrizione alimentare standardizzato per supportare la ricerca, le applicazioni di consumo e quelle industriali.
Al consorzio si sono poi rapidamente unite figure provenienti dall'Universit� della British Columbia, dal British Columbia Center for Disease Control Public Health Laboratory e dalla comunit� \href{http://obofoundry.org}{OBO Foundry} avendo esigenze parallele per l'agricoltura, l'analisi nutrizionale e la ricerca sulle scienze alimentari.
La proposta di FoodOn si basa sulla condivisione e promuove uno standard che nasce dalla congiunzione di domini ontologici coordinati e supportati dalla comunit�, coinvolgendo tassonomie vegetali e animali, nomi comuni, anatomia e terminologia di descrizione di alimenti per umani e animali domestici, riducendo cos� il costo e il carico di lavoro su ogni singolo implementatore.
Il vocabolario di FoodOn deriva dalla trasformazione di LanguaL\footnote{LanguaL, che sta per ``\textbf{Langua aL}imentaria", � un dizionario di indicizzazione alimentare avviato alla fine degli anni '70 dal Center for Food Safety and Applied Nutrition (CFSAN) della Food and Drug Administration degli Stati Uniti grazie alla cooperazione di specialisti in tecnologie alimentari, scienze dell'informazione e nutrizione.} in un vocabolario OWL che fornisce interoperabilit� al sistema, controlla di qualit� e intelligenza basata sul software.

\section{Finalit�}

\index{Finalit�}