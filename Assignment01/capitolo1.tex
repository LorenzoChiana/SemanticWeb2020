\chapter{Introduzione}

\section{Motivazione}

\index{Motivazione}

Al giorno d'oggi la costruzione di reti progettate per la condivisione ad alta capacit� di dati e il fatto che la comunicazione e la diffusione di concetti viene maggiormente veicolata attraverso l'uso di computer e di tecnologie simili, hanno permesso lo scambio sempre pi� crescente di dati governativi e commerciali mettendo, per�, in luce un problema fondamentale: la molteplicit� di lingue e di dizionari di dati.\\
Questo problema comporta un blocco del contenuto delle informazioni esistenti e connesse ad Internet e la mancanza di una lingua franca digitale � evidente nel dominio del cibo, poich� i materiali viaggiano dalla loro provenienza, agricola o meno, ai consumatori attraverso catene di processo e di distribuzione.\\
Un vocabolario bene definito e gerarchico, con relazioni logiche, ossia un'ontologia, � necessario per aiutare ad armonizzare quei dati che abbracciano i settori della sicurezza, qualit�, produzione e distribuzione alimentare e, di conseguenza, della salute e convenienza dei consumatori.

\section{Il progetto FoodOn}

\index{Il progetto FoodOn}

\href{https://foodon.org}{FoodOn} � un progetto guidato dal consorzio Hsiao Lab nato per costruire un'ontologia globale \textit{farm-to-fork}, ossia un vocabolario alimentare standardizzato per supportare la sorveglianza di routine e l'analisi delle epidemie di patogeni di origine alimentare.
Al consorzio si sono poi rapidamente unite figure provenienti dal mondo accademico e dalla comunit� \href{http://obofoundry.org}{OBOFoundry} avendo esigenze parallele per l'agricoltura, l'analisi nutrizionale e la ricerca sulle scienze alimentari.
Il progetto, ad ora, risolve alcune lacune nella terminologia dei prodotti alimentari e supporta la tracciabilit� degli alimenti concentrandosi sulla descrizione di alimenti per umani e animali domestici.
Il vocabolario di FoodOn deriva dalla trasformazione di LanguaL\footnote{LanguaL, che sta per ``\textbf{Langua aL}imentaria", � un dizionario di indicizzazione alimentare avviato alla fine degli anni '70 dal Center for Food Safety and Applied Nutrition (CFSAN) della Food and Drug Administration degli Stati Uniti grazie alla cooperazione di specialisti in tecnologie alimentari, scienze dell'informazione e nutrizione.} in un vocabolario OWL che fornisce interoperabilit� al sistema, controlla di qualit� e intelligenza basata sul software.

\section{Finalit�}

\index{Finalit�}