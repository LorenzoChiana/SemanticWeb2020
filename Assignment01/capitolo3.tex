\chapter{Conclusioni}
Questo progetto, che \`e in continua evoluzione, porta con s\'e un nuovo modello standard per la rappresentazione del cibo che spazia da aspetti come la fonte, lo stato fisico e la forma ad aspetti come l'origine culturale, processi di cottura, conservazione e trattamento, gruppo di consumatori e tipologia di prodotto.
La sua possibilit\`a di integrarsi con altri vocabolari semantici si rivela uno strumento chiave nel monitoraggio dei flussi di risorse e rifiuti tra sistemi antropogenici e naturali soprattutto nell'attuale era di crescente globalizzazione delle reti alimentari.
Inoltre, FoodOn si interessa anche dei recenti temi ambientali dati gli impatti del sistema alimentare umano sulla biodiversit\`a e sui servizi eco-sistemici. Per questo \`e stato progettato anche per essere interoperabile con ontologie incentrate sulla conoscenza ecologica, le pratiche agronomiche e lo sviluppo sostenibile.\\

In breve: l'impegno congiunto tra la comunit\`a universitaria e quella scientifica per la sua creazione ha donato a questo progetto solide basi.
Da queste, l'essere open-source e la possibilit\`a di essere integrato con ontologie di terze parti porta ad un costante supporto della comunit\`a che, molto probabilmente, lo porter\`a ad essere un progetto longevo e sempre al passo coi tempi.