\chapter{Panoramica}
\section{Cos'� un sistema ADAS}
Il continuo progresso tecnologico ha portato, negli ultimi anni, le automobili moderne in veri e propri sistemi elettronici dotati di assistenza alla guida.
Questi ausili elettronici sono indicati con l'acronico ADAS auto (Advanced Driver Assistance Systems) e, nonostante siano stati sviluppati principalmente per tutelare l'incolumit� del guidatore e passeggero, con questa sigla si identificano anche tutti quei dispositivi presenti nell'auto per incrementare il comfort di guida.
\section{Sicurezza}
Come gi� accennato l'obiettivo principale di un sistema ADAS � la sicurezza che si tramuta in riduzione del rischio di incidente grazie a diversi sistemi di controllo.
Controlli che vanno dall'avviso di collisione a quelli di velocit�.
\section{Perch� le ontologie?}
Gli attuali veicoli a guida autonoma in fase di sviluppo sono dotati di diversi sensori altamente sensibili come camera, stereo camera, Lidar, and Radar.
Sebbene oggetti e corsie possano essere rilevati utilizzando questi sensori, i veicoli non possono comprendere il significato degli ambienti di guida senza la rappresentazione della conoscenza dei dati.
Pertanto un metodo di rappresentazione della conoscenza comprensibile da una macchina pu� essere una soluzione pi� che necessaria per colmare il divario tra gli ambienti di guida rilevati e l'elaborazione della conoscenza.
Le ontologie vengono in aiuto in quanto sono le framework strutturali per la rappresentazione della conoscenza sul mondo o su una parte di esso, che � composto principalmente di concetti (classi) e relazioni (propriet�) tra essi.
\section{Scopo progetto}
Questo progetto ha lo scopo di studiare, comprendere ed estendere l'ontologia gi� fornita da Ichise Laboratory.
Si pone l'obiettivo di raggiungere i seguenti requisiti:
\begin{itemize}
	\item regolazione velocit� in presenza di segnaletica di pericolo;
	\item gestione azioni di fronte a un semaforo.
\end{itemize}
Inoltre ha lo scopo di utilizzare il reasoner OWLAPI mediante linguaggio Java.
