\chapter{Conclusioni}
Questo progetto, che \`e in continua evoluzione, porta con s\'e un nuovo modello standard per la rappresentazione del cibo a livello mondiale che spazia da aspetti come la fonte, lo stato fisico e la forma; ad aspetti come l'origine culturale, processi di cottura, conservazione e trattamento, gruppo di consumatori e tipologia di prodotto.
La sua possibilit\`a di integrarsi con altri vocabolari semantici si rivela uno strumento chiave nel monitoraggio dei flussi di risorse e rifiuti tra sistemi antropogenici e naturali soprattutto nell'attuale era di crescente globalizzazione delle reti alimentari.
Inoltre, FoodOn si interessa anche dei recenti temi ambientali dati gli impatti del sistema alimentare umano sulla biodiversit\`a e sui servizi eco-sistemici. Per questo \`e stato progettato anche per essere interoperabile con ontologie incentrate sulla conoscenza ecologica, le pratiche agronomiche e lo sviluppo sostenibile.\\
Per le future release si prevede che la struttura tassonomica di classificazione di FoodOn sar\`a ulteriormente sviluppata per supportare la classificazione intuitiva degli alimenti e delle loro sfaccettature mantenendo, comunque, la coerenza con le ontologie correlate.\\
Un problema che l'attuale stato del progetto porta con s\'e riguarda ad alcune sfaccettature di LanguaL che contengono termini che possono essere ulteriormente definiti utilizzando una combinazione di nuove classi e classi gi\`a esistenti di altre ontologie.
Alcuni termini di FoodOn, probabilmente, saranno trasferiti ad altre ontologie consentendo, di conseguenza, di descrivere i database alimentari indicizzati con LanguaL gi\`a esistenti con componenti a maggior granularit\`a e di consentire a ragionatori di riconoscere membri appropriati di un tipo di prodotto alimentare e possibili componenti alimentari.
La granularit\`a per FoodOn \`e ancora un problema, basti pensare, ad esmpio, che la FAO elenca pi\`u di 12000 specie di pesci e molti di questi in FoodOn devono ancora essere importati e collegati all'uso culturale regionale.
Una possibile soluzione di questo problema per le prossime versioni del progetto \`e quella di incorporare le risorse che utilizzano gli standard del programma INFOODS\footnote{INFOODS  \`e una rete mondiale di esperti di composizione degli alimenti che mira a migliorare la qualit\`a, la disponibilit\`a, l'affidabilit\`a e l'uso dei dati sulla composizione degli alimenti.} della FAO.\\
Inoltre, in progetto c'\`e anche la possibilit\`a di andare ad esplorare la fattibilit\`a di un portale per la cura della terminologia, simil Wikipedia, che consente alle persone di suggerire definizioni o collegamenti a definizioni applicabili, sinonimi, riferimenti di tassonomia e riferimenti di immagini per termini per garantire un supporto migliore al multilingua e alla diversit\`a culturale.
Le varie proposte verranno poi supervisionate da un team che poi eventualmente le approver\`a.\\
La Hsiao Lab \`e, per ora, il principale curatore di FoodOn, ma l'ambizione globale del progetto dipende dall'attrazione diretta di nuovi partner e di una rete estesa di gruppi di lavoro IC3-FOODS\footnote{International Center for Food Ontology Operability Data and Semantics, consorzio internazionale che collabora con l'industria e i partner accademici di tutto il mondo per costruire una piattaforma semantica per l'IoF (Internet of Food).} attualmente in fase di organizzazione.\\
La struttura e il design di base del progetto ormai sono gi\`a stabili e ben definiti e ci\`o consente la partecipazione di organizzazioni private o pubbliche per aiutare a guidare, curare e fornire feedback sul suo sviluppo.
Attualmente il progetto FoodOn partecipa alla nuova discussione sull'armonizzazione dei sistemi di classificazione degli alimenti delle agenzie del gruppo di lavoro GODAN\footnote{Global Open Data for Agriculture and Nutrition o GODAN \`e un consorzio senza scopo di lucro finanziato da ONG e agenzie governative per l'alimentazione e la nutrizione.}.
Da ci\`o il progetto prevede l'emergere di un piano di sviluppo attuabile a lungo termine che includa sia i finanziamenti delle sovvenzioni stimolati dalla possibile diffusione del progetto, sia un modello di governance supportato dalla partecipazione a livello di agenzia.
