\chapter{Panoramica}

\section{Cos'� un sistema ADAS}

\index{Cos'� un sistema ADAS}

Il continuo progresso tecnologico ha portato, negli ultimi anni, le automobili moderne a diventare veri e propri sistemi elettronici dotati di assistenza alla guida.
Questi ausili elettronici sono indicati con l'acronico ADAS auto (Advanced Driver Assistance Systems) e, nonostante siano stati sviluppati principalmente per tutelare l'incolumit� del guidatore e passeggero, con questa sigla si identificano anche tutti quei dispositivi presenti nell'auto per incrementare il comfort di guida.\cite{automobileit}

\section{Sicurezza}
\index{Sicurezza}

Come gi� accennato l'obiettivo principale di un sistema ADAS � la sicurezza che si tramuta in riduzione del rischio di incidente grazie a diversi sistemi di controllo.
Controlli che vanno dall'avviso di collisione a quelli di velocit�.\cite{automobileit}

\section{Perch� le ontologie?}
\index{Sicurezza}

Gli attuali veicoli a guida autonoma in fase di sviluppo sono dotati di diversi sensori altamente sensibili come camera, stereo camera, Lidar, and Radar.
Sebbene oggetti e corsie possano essere rilevati utilizzando questi sensori, i veicoli non possono comprendere il significato degli ambienti di guida senza la rappresentazione della conoscenza dei dati.
Pertanto un metodo di rappresentazione della conoscenza comprensibile da una macchina pu� essere una soluzione pi� che necessaria per colmare il divario tra gli ambienti di guida rilevati e l'elaborazione della conoscenza.
Le ontologie vengono in aiuto in quanto sono i framework strutturali per la rappresentazione della conoscenza sul mondo o su una parte di esso, che � composto principalmente di concetti (classi) e relazioni (propriet�) tra essi.

\section{Scopo progetto}
\index{Scopo progetto}
Questo progetto si pone diversi obiettivi:
\begin{itemize}
\item studiare, comprendere ed estendere l'ontologia gi� fornita da \href{http://ri-www.nii.ac.jp/index.html}{Ichise Laboratory};
\item raggiungere il seguente requisito: regolazione velocit� al superamento dei limiti di velocit�;
\begin{comment}
	\begin{itemize}
		\item regolazione velocit� al superamento dei limiti di velocit�;
		\item gestione azioni in presenza di semaforo.
	\end{itemize}
\end{comment}
\item utilizzare \href{http://owlapi.sourceforge.net/}{OWL API} mediante linguaggio Java.
\item creare una simulazione dove il veicolo procede con una certa direzione rispettando i limiti di velocit\`a imposti e testare il tutto.
\end{itemize}

\section{Analisi}
\subsection{Regolazione velocit� per il rispetto del limite stradale}
\index{Requisito 1}
Con questo requisito si vuole andare a modellare uno scenario in cui lungo la carreggiata il veicolo debba modificare la propria andatura per il rispetto dei limiti di velocit� imposti dal regolamento stradale.
Come si potr� notare nel prossimo capitolo (\ref{CarOnto}) un veicolo � in grado di modificare la propria andatura accelerando, decelerando o mantenendo l'andatura. Di conseguenza di vuole sfruttare ci� per modellare uno scenario dove un veicolo possa accelerare fino a una soglia massima che � il limite di velocit� imposto, da l� in poi potr� solo procedere a velocit� costante o decelerare.

\begin{comment}
\subsection{Gestione azioni in presenza di semaforo}
\index{Requisito 2}
Con questo requisito, invece, si vuole modellare uno scenario in cui il veicolo, dotato di sistema ADAS, si trova ad attraversare un passaggio con presenza semaforica.
In questo caso dovr� compiere azioni differenti a seconda del proprio stato e di quello del semaforo:
\begin{itemize}
	\item se il veicolo � in moto e il semaforo � verde potr� continuare;
	\item se il veicolo � fermo e il semaforo � verde dovr� ripartire;
	\item se il veicolo � in moto e il semaforo � arancione dovr� rallentare;
	\item se il semaforo � rosso dovr� fermarsi.
\end{itemize}
\end{comment}