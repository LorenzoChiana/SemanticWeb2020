\chapter{Implementazione}

\section{Modellazione entit\`a}\label{modellazione}
Di seguito verr\`a esposto come le varie entit\`a sono state modellate all'interno del progetto.
Queste vanno a modellare e a inserire nell'ontologia i seguenti concetti:
\begin{itemize}
\item il mio veicolo con sistemi ADAS (MyCar);
\item il percorso del veicolo (Map);
\item i controlli sui vari segmenti (SegmentControl);
\item il segmento di strada (RoadSegment);
\item il profilo della velocit\`a (SpeedProfile).
\end{itemize}
\subsection{MyCar + MyCarUtils}
Attraverso la classe singleton \textit{MyCar} viene modellato il concetto di automobile con sistemi ADAS. \`E stata creata riutilizzando la classe ontologica \textit{MyCar} di \textit{CarOnto} e le propriet\`a \textit{isRunningOn} e \textit{carID} (vedi propriet\`a riutilizzate in CarOnto \ref{propCarOnto}).
Insieme alla classe \textit{MyCar} \`e stata introdotta anche la relativa classe di utility \textit{MyCarUtils} che permette, attraverso il metodo \textit{addMyCar}, l'aggiunta di un nuovo individual della classe \textit{myCar} all'interno dell'ontologia che rappresenta l'automobile con sistemi ADAS.
Inoltre, attraverso il metodo \textit{connectPropertiesToMyCar} si aggiunge il data property carID al veicolo e col metodo \textit{setMyCarPositionAndStartRunning} viene assegnata all'auto l'object property \textit{isRunningOn} che specifica che \`e in movimento.
\subsection{SegmentControl + SegmentControlUtils}
Attraverso la classe singleton \textit{SegmentControl} viene modellato il concetto di controlli sui vari segmenti di strada. Nello specifico, questi controlli sono inerenti a:
\begin{itemize}
\item corsia di inizio e fine percorso, tramite il riutilizzo delle classi \textit{StartLane} ed \textit{EndLane} di \textit{ControlOnto};
\item direzione della corsa, tramite la classe \textit{GoForward} dato che il percorso modellato non presenter\`a nessuna deviazione;
\item avvertimento del superamento dei limiti di velocit\`a, attraverso il riutilizzo della classe \textit{OverSpeedWarning}.
\end{itemize}
Le proprit\`a riutilizzate saranno \textit{nextPathSegment} e \textit{pathSegmentID} (vedi propriet\`a riutilizzate in ControlOnto \ref{propControlOnto}), pi\`u la nuova object property \textit{overSpeedWarningThan} (\ref{estensione})).
Insieme alla classe \textit{SegmentControl} \`e stata introdotta anche la relativa classe di utility \textit{SegmentControlUtils} che permette la creazione di una sequenza di segmenti di strada attraverso il metodo \textit{createPath} nella quale si va a stabilire qual \`e la corsia di inizio e quale quella di fine del percorso.
Ad ogni segmento passatogli in input che andr\`a a comporre la rotta da percorrere verr\`a assegnato il data property \textit{pathSegmentID} e create le varie triple \textit{Segment\_i - nextPathSegment - Segment\_i+1}.
Infine, verr\`a posizionata la vettura con sistemi ADAS sul primo segmento che andr\`a a comporre il percorso.
Inoltre, in questa classe di utility sono presenti i metodi \textit{addClassForSpeedAndGoForwardSWRLRules} e \textit{connectPropertiesForSpeedSWRLRules} che aggiungono le classi e propriet\`a per le regole SWRL riguardanti i limiti di velocit\`a e la direzione di guida.
\subsection{Map + MapUtils}
Attraverso la classe singleton \textit{Map} viene modellato il concetto di percorso.
Sono state riutilizzate le classi \textit{RoadSegment}, \textit{OneWayLane}, \textit{SpeedLimit} e delle propriet\`a \textit{hasLane}, \textit{isConnectedTo}, \textit{isLaneOf}, \textit{goStraightTo}, \textit{speedMax} (si vedano le sezioni \ref{riutilizzoclassi} \ref{propMapOnto}).
Inoltre, tale classe contiene i due segmenti di strada di inizio e fine di tipo \textit{RoadSegment}(sezione \ref{roadsegment}).
Insieme alla classe \textit{Map} \`e stata introdotta anche la relativa classe di utility \textit{MapUtils} che permette, attraverso il metodo \textit{addRoadSegment}, l'aggiunta due nuovi segmenti di strada: uno di inizio e uno di fine.
Questi segmenti di strada vengono creati aggiungendo all'ontologia nuovi individual delle classi \textit{RoadSegment} e \textit{OneWayLane} per realizzare un segmento composto da due corsie a senso unico.
Con il metodo \textit{connectObjectPropertiesToRoadSegment} si va a definire:
\begin{itemize}
\item la proprit\`a \textit{hasLane} \`e inversa rispetto a \textit{isLaneOf};
\item le asserzioni \textit{RoadSegment - hasLane - LaneRight} e \textit{RoadSegment - hasLane - LaneLeft};
\item l'asserzione \textit{LaneRight of RoadSegmentStart - goStraightTo - LaneRight of RoadSegmentStop}.
\end{itemize}
Il metodo \textit{addSpeedLimit} permette l'aggiunta nell'ontologia di un nuovo individual della classe SpeedLimit che rappresenta i limiti di velocit\`a e, attraverso \textit{connectSpeedMaxProperty}, si va ad assegnargli il data property \textit{speedMax} settato a 50, modellando cos\`i un limite di velocit\`a impostato a 50km/h.
\subsection{RoadSegment}\label{roadsegment}
RoadSegment \`e una classe che modella un segmento di strada ed \`e composta da tre individual:
\begin{itemize}
\item road, il segmento;
\item laneRight, la corsia di destra di cui il segmento \`e composto;
\item laneLeft, la corsia di sinistra di cui il segmento \`e composto.
\end{itemize}
\subsection{SpeedProfile + SpeedProfileUtils}
Attraverso la classe singleton SpeedProfile viene modellato il concetto del profilo della velocit\`a che un veicolo pu\`o sostenere, ovvero: accelerazione, decelerazione e velocit\`a costante.
Vengono riutilizzate le classi \textit{Acceleration}, \textit{ConstantSpeed} e \textit{Deceleration} di \textit{CarOnto} (sezione \ref{riutilizzoclassi}).
Insieme alla classe \textit{SpeedProfile} \`e stata introdotta anche la relativa classe di utility \textit{SpeedProfileUtils} che permette, attraverso il metodo \textit{addSpeedLimit}, di settare le classi ontologiche dei tre profili di velocit\`a.
\section{Altre classi di utility}
Per sviluppare questo progetto sono state create diverse classi di utility tra cui:
\begin{itemize}
\item \textit{\textbf{OntologyUtils}} che contiene i diversi metodi per la creazione di un'ontologia OWL e delle sue parti come: la creazione di una nuova ontologia e, data un'ontologia, l'aggiunta di classi, individual, object property, data property, assiomi e cos\`i via;
\item \textit{\textbf{OWLOntologyUtils}} che contiene vari metodi per la gestione di un'ontologia OWL quali: l'aggiunta e la rimozione di assiomi, l'importazione di ontologie remote o ottenere il data factory dell'ontologia che pu\`o essere utilizzata in seguito per creare oggetti OWLAPI come classi, propriet\`a, individual, assiomi ecc.
\item \textit{\textbf{SWRLUtils}} contiene i metodi per la creazione di regole, classi, object property e variabili SWRL;
\item \textit{\textbf{ReasonerUtils}} contiene il metodo per la creazione del reasoner HermiT e i metodi per ricavare la posizione di myCar e i nodi di inizio e fine;
\item \textit{\textbf{SimulationUtils}} contiene i metodi per la creazione della simulazione che verr\`a approfondita nella sezione seguente;
\item \textit{\textbf{IRIs}} contiene i vari IRI utilizzati all'interno del progetto;
\end{itemize}

\section{La simulazione - SimulationUtils}
La simulazione consiste nella creazione di una strada composta da due segmenti a due corsie a senso unico nella quale viene posizione un veicolo.
Questo veicolo viene modellato in modo tale da non superare i limiti di velocit\`a, impostati arbitrariamente a 50km/h, e deve procedere dal primo segmento al secondo mantenendo la corsia di destra.
Attraverso il metodo \textit{createSimulation} di \textit{SimulationUtils} viene creata tale simulazione inizializzando una nuova ontologia vuota con \textit{createMyOntology} e i vari singleton attraverso il metodo privato \textit{initClasses}.
Mediante il metodo \textit{createMyCarRoute} viene creato il percorso formato da due segmenti di strada.
Tutti questi metodi fanno uso delle classi singleton e di utility esposte in dettaglio nella sezione \ref{modellazione} per la creazione della simulazione.
I metodi \textit{createSpeedLimitSWRLRule} e \textit{createRunDirectionRule} vanno a creare le regole SWRL rispettivamente per la gestione dei limiti di velocit\`a e della direzione di corsa che sono esposte in dettaglio nella sezione \ref{SWRLspeedlimit} e \ref{SWRLrundirection}.
\subsection{Regole SWRL per la gestione dei limiti stradali}\label{SWRLspeedlimit}
All'interno del metodo \textit{createSpeedLimitSWRLRule} vengono implementare le regole generali per il monitoraggio della velocit\`a. Nello specifico vengono implementate le due regole portare in esempio nella sezione dedicata a SWRL (\ref{SWRL}), ovvero:
\begin{center}
\emph{isRunningOn(?X, ?Lane) $\char`\^$ OneWayLane(?Lane) $\char`\^$ SpeedLimit(?Y) $\char`\^$ overSpeedWarningThan(?X, ?Y) $\rightarrow$ constantSpeed(?X)}\\
\end{center}
per la gestione del profilo di velocit\`a quando viene segnalato l'avviso di eccesso dei limiti e
\begin{center}
\emph{isRunningOn(?X, ?Lane) $\char`\^$ OneWayLane(?Lane) $\rightarrow$ acceleration(?X)}\\
\end{center}
invece nel caso in cui non venga segnalato l'eccesso dei limiti.

Infine, \`e stata creata la classe di test \textit{SpeedLimitTest} nella quale sono stati implementati due differenti test:
\begin{enumerate}
\item test della simulazione senza avviso del superamento dei limiti;
\item test della simulazione con avviso del superamento dei limiti.
\end{enumerate}
\subsection{Regole SWRL per la gestione della direzione}\label{SWRLrundirection}
All'interno del metodo \textit{createRunDirectionRule} viene implementata la regola generale per la gestione della direzione del veicolo che, in questo caso specifico, sar\`a quella di proseguire dritto.
Tale regola \`e:
\begin{center}
isRunningOn(?X, ?Lane) $\char`\^$ goStraightTo(?Lane, ?NextLane) $\char`\^$ nextPathSegment(?Lane, ?NextLane) $\rightarrow$ GoForward(?X)
\end{center}

Infine, \`e stata creata la classe test \textit{GoForwardTest} che va a controllare la direzione dell'avanzamento dell'automobile e che i nodi di inizio e fine combacino rispettivamente con i segmenti di strada di inizio e fine. 
